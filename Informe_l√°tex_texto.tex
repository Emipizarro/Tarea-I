\documentclass{article}
\usepackage{graphicx} % Required for inserting images

\title{\textbf{Informe}}
\author{Emilia Pizarro}
\date{12 Mayo 2025}

\begin{document}

\maketitle

\section{Algoritmo}
\begin{figure}[h]
    \centering
    \includegraphics[width=1.\linewidth]{criba_eratostenes.jpg}
    \caption{Algoritmo Eratóstenes}
    \label{fig:enter-label}
\end{figure}
Este Algoritmos lo aprendimos en clase, y luego tuvimos la oportunidad de implementarlo en uno de los sets de Introducción a la programación
\newpage
\section{Autor}
\begin{figure}[h]
    \centering
    \includegraphics[width=0.5\linewidth]{20.jpg}
    \caption{Ertóstenes}
    \label{fig:enter-label}
\end{figure}
  El creador de este algoritmo es Eratóstenes, matemático y astrónomo griego de los años 200 A.C.
  \newline
  Su logro más reconocido, calcular el diamétro y la circunferencia de la tierra. Además de ser la primera persona en lograr calcular, y con tal presición dicho acontecimiento, dio raices al algoritmo que porta su propio nombre.
  \newline
  El algoritmo de la criba de eratóstenes nos permite encontrar todos los números primos hasta un número dado.

\newpage
\section{Uso}
\begin{figure}[h]    
\centering
    \includegraphics[width=1\linewidth]{Captura de pantalla 2025-05-12 212352.png}
    \caption{Algoritmo en ejecución hoy en día}
    \label{fig:enter-label}
\end{figure}
Su funcionamiento es bastante simple, se aplica una serie de comadnos, que empleando el menor número de una lista que empieza en 2, va eliminando todos aquellos números hasta n, que son divisibles por el mismo.
\newline
De esta manera, rápidamente se logra acortar la longitud de la lista inicial, y finalmente, llegar a los números primos contenidos en la serie de números hasta uno entregado.

\end{document}
